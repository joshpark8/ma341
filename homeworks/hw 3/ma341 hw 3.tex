\documentclass[11pt]{article}

% Packages
\usepackage[margin=1in]{geometry}
\usepackage{amsfonts}
\usepackage{amsmath}
\usepackage{amssymb}
\usepackage{amsthm}
\usepackage{fleqn}

% Fundamental sets
\newcommand{\N}{\mathbb{N}}    % Natural
\newcommand{\R}{\mathbb{R}}    % Real
\newcommand{\C}{\mathbb{C}}    % Complex
\newcommand{\Q}{\mathbb{Q}}    % Rational
\newcommand{\Z}{\mathbb{Z}}    % Integers

% Title
\author{Josh Park}
\date{Spring 2024}
\title{MA 34100 Homework 3}

% Document
\begin{document}
\setlength{\parindent}{0pt}
\maketitle
\section*{Section 2.2}
\subsection*{2.2.2) If $a, b \in \R$, show that $\vert a+ b \vert = \vert a \vert + \vert b \vert$ \emph{iff} $ab \geq{ 0}$}
    To show this, we must prove both directions of the biconditional statement.
    \subsubsection*{1) Assume $ab \geq 0$ is true.}
    This means that either $a$ and $b$ are both non-negative, or both non-positive. Then, we can split the proof into two cases.
    \begin{enumerate}
        \item[(i)] If $a$ and $b$ are both non-negative, $a+b$ is necessarily non-negative, and $\vert a+b \vert = a+b$. Because they are non-negative, we know $|a|=a$ and $|b|=b$, so it follows that $|a|+|b| = |a+b|$ is true.

        \item[(ii)] If $a$ and $b$ are both non-positive, notice that $\vert a+b \vert = -(a+b)$. Also, it follows that $|a|=-a$ and $|b|=-b$, so $|a|+|b| = |a+b|$ is true.
    \end{enumerate}
    \subsubsection*{2) Assume $|a|+|b| = |a+b|$ is true.}
    From this, we know that $a+b$ does not change signs, so $a$ and $b$ must be both non-positive or non-negative. If $a$ and $b$ have different signs, we know $|a+b|$ must be less than the sum of $|a|$ and $|b|$, because the negative and positive would partially cancel each other. This contradicts our assumption that $|a|+|b| = |a+b|$, so $a$ and $b$ must have the same sign. We have now shown that both directions of the biconditional statment are dependant on each other \qed

\subsection*{2.2.9) Find all values that satify the following inequalities.}
    \subsubsection*{a) $\qquad |x-2| \leq x+1$}
        For the first case, assume $x \leq 2$. Then, $|x-2|=2-x$, and the inequality becomes
        \begin{flalign} 2-x \leq x+1 \implies \frac{1}{2}\leq x. \end{flalign}
        Thus, case one gives us $ \frac{1}{2}\leq x \leq 2$. For the second case, assume $x \geq 2$. Then, $|x-2|=x-2$ and the inequality becomes
        \begin{flalign} x-2 \leq x+1. \end{flalign}
        This simplifies to be trivially true, so the solution set is $x \in[\frac{1}{2}, \infty)$.

    \subsubsection*{b) $\qquad 3|x| \leq 2-x$}
        For the first case, assume $x \geq 0$. Then, $3|x|=3x$, and the inequality becomes
        \begin{flalign} 3x \leq x-2 \implies x \leq -1. \end{flalign}
        This contradicts our assumption, so $x \geq 0$ must not be possible. Then, for the second case, assume $x < 0$. Then, $3|x|=-3x$ and the inequality becomes
        \begin{flalign} -3x \leq x-2 \implies x \geq \frac{1}{2}. \end{flalign}
        This gives us that $0 \geq x \geq \frac{1}{2}$, so the expression is valid for all $x \in[0, \frac{1}{2}]$.

\subsection*{2.2.10) Find all $x \in \R$ that satisfy the following inequalities.}
    \subsubsection*{a) $\qquad |x-1| > |x+1|$}
        For the first case, consider $x > 1$. Then, $|x-1|=x-1$ and $|x+1|=x+1$, so the inequality becomes
        \begin{flalign} x-1 > x+1. \end{flalign}
        This is trivially false, so $x > 1$ is not a valid solution. For the second case, consider $x < 1$. Then, $|x-1|=1-x$ and $|x+1|=1+x$, so the inequality becomes
        \begin{flalign} 1-x > 1+x. \end{flalign}
        This simplifies to $-x > x$, which is true for all $x < 0$. Thus, the solution set is $x \in(-\infty, 1)$.

    \subsubsection*{b) $\qquad |x|+|x+1| < 2$}
        For the first case, consider $x \geq 0$. Then, $|x|=x$ and $|x+1|=x+1$, so the inequality becomes
        \begin{flalign} x+x+1 < 2 \implies x < \frac{1}{2}. \end{flalign}
        This gives us that $x \in[0, \frac{1}{2})$. For the second case, consider $x < 0$. Then, $|x|=-x$ and $|x+1|=-(x+1)$, so the inequality becomes
        \begin{flalign} -x-x-1 < 2 \implies x > -\frac{3}{2}. \end{flalign}
        This gives us that $x \in(-\infty, 0)$. Thus, the solution set is $x \in(-\infty, \frac{1}{2}]$.

\subsection*{2.2.12) Find all $x \in \R$ that satisfy the inequality $4 < |x+2|+|x-1| < 5$.}
    For the first case, consider $x < -2$, so $|x+2|=-x-2$ and $|x-1|=-x+1$. Then, the inequality becomes
    \begin{flalign} 4 < -x-2-x+1 = -2x-1 < 5 \implies -3 < x < -\frac{5}{2}. \end{flalign}
    This gives that $x \in(-3, -\frac{5}{2})$. For the second case, consider $-2 \leq x < 1$, so $|x+2|=x+2$ and $|x-1|=-x+1$. Then, the inequality becomes
    \begin{flalign} 4 < x+2-x+1 = 3 < 5. \end{flalign}
    This is trivially false, so $-2 \leq x < 1$ is not a valid solution. For the third case, consider $x \geq 1$, so $|x+2|=x+2$ and $|x-1|=x-1$. Then, the inequality becomes
    \begin{flalign} 4 < x+2+x-1 = 2x+1 < 5 \implies \frac{3}{2} < x < 2. \end{flalign}
    This gives that $x \in(\frac{3}{2}, 2)$. Thus, the solution set is $x \in(-3, -\frac{5}{2}) \cup (\frac{3}{2}, 2)$.

\section*{Section 2.3}
\subsection*{2.3.5) Find the infimum and supremum, if they exist, of the following sets.}
    \subsubsection*{a) $\qquad A := \{x \in \R : \; 2x-5 > 0\}$}
        $\inf{A}=-\frac{5}{2}$, and $\sup{A}$ does not exist.
    \subsubsection*{b) $\qquad B := \{x \in \R : \; x+2 \geq x^2\}$}
        $\inf{B}=-2$, and $\sup{B}=2$.
    \subsubsection*{c) $\qquad C := \{x \in \R : \; x < 1/x\}$}
        $\inf{C}=-1$, and $\sup{C}=1$.
    \subsubsection*{d) $\qquad D := \{x \in \R : \; x^2-2x-5 < 0\}$}
        $\inf{D}=1-\sqrt{6}$, and $\sup{D}=1+\sqrt{6}$.

\subsection*{2.3.6) Let $S$ be a nonempty subset of $\R$ that is bounded below. \\ Prove that $\inf{S}=-\sup{\{-s:s \in S\}}$}
    If $S$ is a nonempty subset of $\R$ that is bounded below, then we know that $\inf{S}$ exists. Then, we can define the set $T:=\{-s:s \in S\}$. Obviously, $\inf{S} \in S$, so $-\inf{S}$ is in $T$. By definition of infimum, there are no smaller elements in $S$, so $-\inf{S}$ is the largest element in $T$. Thus, $-\inf{S}=\sup{T}$. Then, by definition of $T$, we know that $\sup{T}=\sup{\{-s:s \in S\}}$, so $\inf{S}=-\sup{\{-s:s \in S\}}$ \qed

\subsection*{2.3.10) \quad Show that if $A$ and $B$ are bounded subsets of R, then $A\cup B$ is a bounded set. Show that $\sup(A\cup B)=\sup\{\sup A, \sup B\}$.}
    If $A$ and $B$ are bounded subsets of $\R$, then we know that $\sup{A}$ and $\sup{B}$ exist. Then, we can define the set $C:=A\cup B$. By definition of supremum, we know that $\sup{A}$ and $\sup{B}$ are the least upper bounds of $A$ and $B$, respectively. Then, by definition of least upper bound, we know that $\sup{A}$ and $\sup{B}$ are greater than or equal to all elements in $A$ and $B$, respectively. Thus, $\sup{A}$ and $\sup{B}$ are greater than or equal to all elements in $C$. Then, by definition of supremum, we know that $\sup{A}$ and $\sup{B}$ are the least upper bounds of $C$. Thus, $\sup{A\cup B}=\sup\{\sup{A}, \sup{B}\}$ \qed

\subsection*{2.3.11) \quad Let $S$ be a bounded set in $\R$ and let $S_0$ be a nonempty subset of $S$. Show that $\inf{S} \leq \inf{S_0} \leq \sup{S_0} \leq \sup{S}$}
    If $S$ is a bounded set in $\R$, then we know that $\inf{S}$ and $\sup{S}$ exist. Then, we can define the set $T:=S_0$. By definition of infimum, we know that $\inf{S}$ is the greatest lower bound of $S$, so it is less than or equal to all elements in $S_0$. Then, by definition of infimum, we know that $\inf{S_0}$ is the greatest lower bound of $S_0$, so it is less than or equal to all elements in $S_0$. Then, by definition of supremum, we know that $\sup{S_0}$ is the least upper bound of $S_0$, so it is greater than or equal to all elements in $S_0$. Then, by definition of supremum, we know that $\sup{S}$ is the least upper bound of $S$, so it is greater than or equal to all elements in $S_0$. Thus, $\inf{S} \leq \inf{S_0} \leq \sup{S_0} \leq \sup{S}$. \qed

\end{document}