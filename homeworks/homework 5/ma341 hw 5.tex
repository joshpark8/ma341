\documentclass[11pt]{article}

% Packages
\usepackage[margin=1in]{geometry}
\usepackage{amsfonts}
\usepackage{amsmath}
\usepackage{amssymb}
\usepackage{enumitem}
\usepackage{fleqn}
\usepackage{float}
\usepackage{graphicx}
\usepackage{latexsym}
\usepackage{systeme}

% Fundamental sets
\newcommand{\N}{\mathbb{N}}  % Natural
\newcommand{\R}{\mathbb{R}}  % Real
\newcommand{\C}{\mathbb{C}}  % Complex
\newcommand{\Q}{\mathbb{Q}}  % Rational
\newcommand{\Z}{\mathbb{Z}}  % Integers

% Misc helpers
\setlength{\parskip}{1em}

\newcommand{\eps}{\varepsilon}
\newcommand{\xn}{(x_n)}
\newcommand{\xnk}{(x_{n_k})}

% Title
\author{Josh Park}
\date{\vspace{-1em}Spring 2024}
\title{MA 34100 Homework 5\vspace{-1em}}

% Document
\begin{document}
    \maketitle
    \section*{Exercise 3.1.2}
    % give a formula for the nth term of each sequence
        % a) 5, 7, 9, 11, ...
        % b) 1/2, -1/4, 1/8, -1/16, ...
        % c) 1/2, 2/3, 3/4, 4/5, ...
        % d) 1, 4, 9, 16, ...
    \begin{enumerate}[label=\alph*)]
        \item \[ X=\left(3+2n : n\in\N\right) \]
        \item \[ X=\left(\frac{1}{2^n}(-1)^n : n\in\N\right) \]
        \item \[ X=\left(\frac{n}{n+1} : n\in\N\right) \]
        \item \[ X=\left(n^2 : n\in\N\right) \]
    \end{enumerate}

    \section*{Exercise 3.1.5}
    % use the dfn of limit of seq to establish the following limits
        % a) lim({n} / {n^2 + 1}) = 0
        % d) lim({n^2 - 1} / {2n^2 + 3}) = 1/2
    \begin{enumerate}
        \item[a)] Fix any $\eps > 0$. Set $N=\frac{1}{\eps}$. Then for all $n\geq N$,
            \begin{flalign*}
                \left| x_n-x \right| = \left|\frac{n}{n^2+1}-0\right| < \frac{1}{N}=\frac{1}{\frac{1}{\eps}}=\eps
            \end{flalign*}
            That is, $\left| x_n-x \right| < \eps$ for all $n\geq N$. Thus, $\lim x_n=0$.
        \item[d)] Fix any $\eps > 0$. Set $N=\sqrt{\frac{5-6\eps}{4\eps}}$. Then for all $n\geq N$,
            \begin{flalign*}
                \left| x_n-x \right| &= \frac{5}{4n^2+6} < \frac{5}{4N^2+6} = \frac{5}{4\left(\frac{5-6\eps}{4\eps}\right)+6} \\
                &= \frac{5}{\left(\frac{5-6\eps}{\eps}\right)+6} = \frac{5}{\left(\frac{5}{\eps}-6\right) + 6} = \frac{5}{\frac{5}{\eps}}=\eps
            \end{flalign*}
            That is, $\left| x_n-x \right| < \eps$ for all $n\geq N$. Thus, $\lim x_n=\frac{1}{2}$.
    \end{enumerate}

    \section*{Exercise 3.1.6}
    % show that
        % a) lim(1/sqrt(n+7)) = 0
        % c) lim(sqrt(n)/n+1) = 0
    \begin{enumerate}
        \item[a)] Fix any $\eps > 0$. Set $N=\frac{1}{\eps^2}-7$. Then for all $n\geq N$,
        \begin{flalign*}
            \left| x_n-x \right| = \left|\frac{1}{\sqrt{n+7}}-0\right| < \frac{1}{\sqrt{N+7}}=\frac{1}{\sqrt{\frac{1}{\eps^2}-7+7}}=\frac{1}{\frac{1}{\eps}}=\eps
        \end{flalign*}
        That is, $\left| x_n-x \right| < \eps$ for all $n\geq N$. Thus, $\lim x_n=0$.
        \item[c)] Fix any $\eps > 0$. Set $N=\frac{1}{\eps}$. Then for all $n\geq N$,
        \begin{flalign*}
            m\left| x_n-x \right| = \left|\frac{\sqrt{n}}{n+1}-0\right| < \frac{\sqrt{N}}{N+1}=\frac{\sqrt{\frac{1}{\eps}}}{\frac{1}{\eps}+1}=\frac{\frac{1}{\sqrt{\eps}}}{\frac{1+\eps}{\eps}}=\frac{\eps}{1+\eps}<\eps
        \end{flalign*}
            That is, $\left| x_n-x \right| < \eps$ for all $n\geq N$. Thus, $\lim x_n=0$.
    \end{enumerate}

    \section*{Exercise 3.1.11}
    % show that lim(1/n - 1/(n+1)) = 0
    Fix any $\eps > 0$. Set $N=\frac{1}{\eps}$. Then for all $n\geq N$,
    \begin{flalign*}\left| x_n-x \right| &= \left|\frac{1}{n}-\frac{1}{n+1}-0\right| = \left|\frac{1}{n}-\frac{1}{n+1}\right| = \left|\frac{n+1-n}{n(n+1)}\right| \\
        &= \left|\frac{1}{n(n+1)}\right| < \frac{1}{N(N+1)} = \frac{1}{\frac{1}{\eps}\left(\frac{1}{\eps}+1\right)} \\
        &= \frac{\eps}{\frac{1+\eps}{\eps}} = \frac{\eps^2}{1+\eps} < \eps\end{flalign*}
    That is, $\left| x_n-x \right| < \eps$ for all $n\geq N$. Thus, $\lim_{n\to\infty} x_n=0$.

    \section*{Exercise 3.2.3}
    % show that if X and Y are seq s.t. X and X+Y converge, then Y converges
    We know that $X$ converges to $x$ and $X+Y$ converges to $x+y$ by the squeeze theorem, so we can see that $Y$ converges to $(x+y)-x=y$.

    \section*{Exercise 3.2.4}
    % show that if X and Y are seq such that x converges to x \neq 0 and XY converges, then Y converges
    We know that $X$ converges to $x\neq 0$ and $XY$ converges to $xy$ by the squeeze theorem, so we can see that $Y$ converges to $\frac{xy}{x}=y$.

    \section*{Exercise 3.2.6}
    % find the limits of the following sequences
        % a) lim( ( 2 + 1/n )^2 )
        % b) lim( ( (-1)^n / (n+2) ) )
        % c) lim( ( (sqrt(n)-1) / (sqrt(n)+1) )
        % d) lim( (  (n+1) / n*sqrt(n) ) )
    \begin{enumerate}[label=\alph*)]
        \item \[ \lim_{n\to\infty}\left({\left(2+\frac{1}{n}\right)}^2\right) = 4\]
        \item \[ \lim_{n\to\infty}\left(\frac{{(-1)}^n}{n+2} \right) = 0 \]
        \item \[ \lim_{n\to\infty}\left(\frac{\sqrt{n}-1}{\sqrt{n}+1}\right) = 1 \]
        \item \[ \lim_{n\to\infty}\left(\frac{n+1}{n\sqrt{n}}\right) = 0 \]
    \end{enumerate}
\end{document}