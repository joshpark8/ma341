\documentclass[11pt]{article}

% Packages
\usepackage[margin=1in]{geometry}
\usepackage{amsfonts, amsmath, amsthm, amssymb}
\usepackage{enumitem}
\usepackage{fleqn}
\usepackage{float}
\usepackage{graphicx}
\usepackage{latexsym}
\usepackage{setspace}
\usepackage{systeme}

% Fundamental sets
\newcommand{\N}{\mathbb{N}}  % Natural
\newcommand{\bbR}{\mathbb{R}}  % Real
\newcommand{\C}{\mathbb{C}}  % Complex
\newcommand{\Q}{\mathbb{Q}}  % Rational
\newcommand{\Z}{\mathbb{Z}}  % Integers

\input{/Users/joshpark/purdue/other/latex/macros.tex}

% Misc helpers
\newcommand{\xn}{(x_n)}
\newcommand{\xnk}{(x_{n_k})}

% Title
\author{Josh Park}
\date{\vspace{-1em}Spring 2024\vspace{-1em}}
\title{MA 34100 Homework 7\vspace{-1em}}

\setlength{\parskip}{1em}
\parindent=0em
\onehalfspacing

% Document
\begin{document}
\maketitle
\section*{Exercise 5.2.1}
\begin{enumerate}[label=(\alph*)]
    \item $\begin{aligned} f(x)=\frac{x^2+2x+1}{x^2+1} \qquad (x\in\bbR)\end{aligned}$

    Notice $f$ is a rational function. \\
    Ex 5.2.3(a) gives us that any rational function is continuous for every $x\in\bbR$ for which it is defined. \\
    The domain of $f$ is $\bbR$, so it follows that $f(x)$ is continuous on $\bbR$.

    \item $\begin{aligned} g(x)=\sqrt{x+\sqrt{x}} \qquad (x\geq 0) \end{aligned}$

    Let $f(x)=x$ for all $x\in\bbR$ and $h(x)=\sqrt(x)$ for $x\geq 0$.\\
    It is trivial that $f(x)$ is continuous on $\bbR$, so by Thm 5.2.5(b) $h(x)$ is continuous on $\bbR^+\cup\{0\}$. \\
    We know that $(f+h)(x)=f(x)+h(x)=x+\sqrt{x},$ so from Thm 5.2.2(a) it follows that $f+h$ is continuous on $\bbR^+\cup\{0\}$. \\
    Then Thm 5.2.7 gives that $g(x)=h\circ(f+h)$ is continuous on $\bbR^+\cup\{0\}$. \\

    \item $\begin{aligned} h(x)=\frac{\sqrt{1+\abs{\sin x}}}{x} \qquad (x\neq 0)\end{aligned}$

    Define $\phi(x)=x$ for $x\in\bbR$, $\gamma(x)=\sin(x)$ for $x\in\bbR$, $\mu(x)=1+x$ for $x\in\bbR$, $\chi(x)=\abs{x}$ for $x\in\bbR$, and $\eta(x)=\sqrt{x}$. \\
    Then, we can write
    $\begin{aligned}[t] h=\frac{\eta\circ(\mu\circ(\chi\circ(\gamma\circ\phi)))}{\phi}. \end{aligned}$ \\
    We know $\phi$, $\gamma$, $\mu$, and $\chi$ are continuous over $\bbR$, so by Thm 5.2.7 $\mu\circ(\chi\circ\gamma)$ is continuous over $\bbR$. \\
    We also know that the domain of $\eta$ is $\bbR^+\cup\{0\}$ and the range of $\mu\circ(\chi\circ\gamma)$ is a subset of $\bbR^+\cup\{0\}$. \\
    So, the range of $\eta\circ\mu\circ(\chi\circ(\gamma\circ\phi))$ is $\bbR$. \\
    So, $h(x)$ is continuous on $\bbR\setminus \{0\}$ by Thm 5.2.2(b).

    \item $\begin{aligned} k(x)=\cos\sqrt{1+x^2} \qquad (x\in\bbR)\end{aligned}$

    Let $\alpha(x)=x^2$ for $x\in\bbR$, $\beta(x)=1+x$ for $x\in\bbR$, $\gamma(x)=\sqrt{x}$ for $x\geq 0$, and $\epsilon(x)=\cos(x)$ for $x\in\bbR$. \\
    We know $\alpha$ is continuous on $\bbR$, so from Thm 5.2.7 it follows that $\beta\circ\alpha$ is continuous on $\bbR$. \\
    Notice that the domain of $\gamma$ is $\bbR^+\cup\{0\}$, of which the range of $\beta\circ\alpha$ is a subset. \\
    So, $\gamma\circ(\beta\circ\alpha)$ is continuous on $\bbR$. \\
    Then from Thm 5.2.7 it follows that $\eta\circ(\gamma\circ(\beta\circ\alpha))$ is continuous on $\bbR$.
\end{enumerate}

\section*{Exercise 5.2.3}\vspace{-1em}
    Consider the functions $f$ and $g$ where
    \begin{align}
        &f(x)=\begin{cases}
            1 & \text{ if } x\in\Q \\
            0 & \text{ if } x\in\bbR\setminus\Q
        \end{cases}
        &g(x)=\begin{cases}
            0 & \text{ if } x\in\Q \\
            1 & \text{ if } x\in\bbR\setminus\Q
        \end{cases}
    \end{align}
    Notice they are both discontinuous everywhere on $\bbR$. We can see that
    \begin{align}
        (f+g)(x)=\begin{cases}
            1+0 & \text{ if } x\in\Q \\
            0+1 & \text{ if } x\in\bbR\setminus\Q
        \end{cases} \implies (f+g)(x)=1 \quad \forall x\in\bbR
    \end{align}
    and
    \begin{align}
        fg(x)=\begin{cases}
            0\cdot 1 & \text{ if } x\in\Q \\
            1\cdot 0 & \text{ if } x\in\bbR\setminus\Q
        \end{cases} \implies fg(x)=0 \quad \forall x\in\bbR,
    \end{align}
    so $(f+g)$ and $fg$ are both continuous for any $c\in\bbR$.
    \vspace{-1em}

\section*{Exercise 5.2.7}\vspace{-1em}
    Consider the function
    \begin{align}
        f(x)=\begin{cases}
            1 & \text{ if } x\in\Q \\
            -1 & \text{ if } x\in\bbR\setminus\Q
        \end{cases},
    \end{align}
    where $f$ is known to be everywhere discontinuous on $\bbR$ (and subsequently on [0,1]). Then,
    \begin{align}
        \abs{f(x)}=\begin{cases}
            1 & \text{ if } x\in\Q \\
            1 & \text{ if } x\in\bbR\setminus\Q
        \end{cases}.
    \end{align}
    Notice that $\abs{f(x)}=1$ for all $x\in\bbR$, so it is trivially continuous on [0,1].
    \vspace{-1em}

\section*{Exercise 5.2.8}\vspace{-1em}
    True. Let $c\in\bbR\setminus\Q$. By the density theorem, there exists a sequence of rational numbers $(x_n)$ such that $(x_n)\to c$. Then, $r_n\in\bbR\setminus\Q$. We are given that $f$ and $g$ are both continuous on $\bbR$, so by the Sequential Criterion for continuity:
    \begin{align} &(f(x_n))\to f(c) &(g(x_n))\to g(c).\end{align}
    Then, for all $n\in\N$, we have that $f(x_n)=g(x_n)$. This implies that
    \begin{align} \lim_{n\to\infty}f(x_n) = \lim_{n\to\infty}g(x_n) \quad\implies\quad f(c)=g(c). \end{align}
    Since $c$ is an arbitrary element of $\bbR\setminus\Q$, we can deduce that
    \begin{align} f(c)=g(c) \quad \forall c\in\bbR\setminus\Q\implies f(n)=g(n) \quad \forall n\in\bbR \end{align}
    \vspace{-1em}

\section*{Exercise 5.3.3}\vspace{-1em}
    $f$ is continuous on $I \implies \abs{f}$ is continuous on $I$. \\
    Let $\alpha =\inf|f|(I)$. \\
    By the Maximum-Minimum Thm, there exists some $c\in I$ such that $\abs{f}(c)=\abs{f(c)}=\alpha$. \\
    Now, we conjecture that $\alpha=0$. For contradiction, we assume $\alpha\neq 0$. \vspace{-1em}
    \begin{proof}
        We know that for any $x\in I$, there exists some $y\in I$ with the property that $\abs{f(y)}\leq\frac{1}{2}\abs{f(c)}=\frac{1}{2}m$. \\
        Then, $\alpha\neq 0\implies \alpha>0\implies \abs{f(y)} < m = \inf \abs{f}(I)$. \\
        However this is a contradiction by definition of infimum. \\
        Thus our assumption that $\alpha\neq 0$ must be false.
    \end{proof}\vspace{-1em}
    So, there necessarily exists some $c\in I$ with $f(c)=0$.
    \vspace{-1em}

\section*{Exercise 5.3.17}\vspace{-1em}
    Yes, $f:[0,1]\to\bbR$ is a constant function. \vspace{-1em}
    \begin{proof}
    Suppose for contradiction that $f$ was not a constant function. \\
    Then, there must exist numbers $\alpha, \beta \in[0,1]$ such that $f(\alpha)\neq f(\beta)$. \\
    Without loss of generality, let $f(\alpha)<f(\beta)$. \\
    By the Density Theorem, there exists some $\gamma\in\bbR\setminus\Q$ such that $f(\alpha)<\gamma<f(\beta)$. \\
    By the Intermediate Value Theorem, there exists some $\delta\in[0,1]$ with $f(\delta)=\gamma$. \\
    We know $f$ only produces rational numbers so $\gamma\in\Q$. \\
    However, this contradicts with $\gamma\in\bbR\setminus\Q$. \\
    Thus our assumption that $f$ is not a constnat function must be false.
    Thus $f$ is necessarily a constant function.
    \end{proof}

\section*{Exercise 5.4.2}\vspace{-1em}
    Let $\veps>0$ be given and let $\delta=\frac{\varepsilon}{2}$. \\
    Then for all $x,y\in A$ if $\abs{x-y}<\delta$, then
    \begin{align}
        \abs{f(x)-f(y)} &= \abs{\frac{1}{x^2}-\frac{1}{u^2}} \\
            &= \lt(\frac{y+x}{x^2y^2}\rt)\abs{y-x} \\
            &= \lt(\frac{1}{x^2y}+\frac{1}{xy^2}\rt)\abs{y-x}\\
            &\leq 2\abs{x-u}\\
            &< 2\frac{\varepsilon}{2}=\veps
    \end{align}
    So $f$ is uniformly continuous over $A$. \\
    For $B$, suppose we have sequences $x_n=\frac{1}{n}$ and $y_n=\frac{1}{n+1}$. \\
    It follows then that $\abs{x_n-y_n}\to 0$, but
    \begin{align}
        \abs{f(x_n)-f(y_n)} &= \abs{n^2-n^2+2n+1} \\
            &= \abs{2n+1} \geq 1 \text{ for all $n$}
    \end{align}
    Thus $f$ is not uniformly convergent on $B$.

\section*{Exercise 5.4.5}\vspace{-1em}
    Let $\abs{x-y}<\min\lt\{\delta_f(\frac{\varepsilon}{2}),\delta_g(\frac{\varepsilon}{2})\rt\}$. Then,
    \begin{align}
        \abs{(f+g)(x)-(f+g)(y)} &= \abs{(f(x)+g(x))-(f(y)+g(y))} \\
        &\leq \abs{f(x)+f(y)} + \abs{g(x)+g(y)} \\
        &< \frac{\varepsilon}{2}+\frac{\varepsilon}{2} = \varepsilon
    \end{align}

\section*{Exercise 5.4.6}\vspace{-1em}
    Let $\abs{x-y}<\min\lt\{\delta_f(\frac{\varepsilon}{2M}),\delta_g(\frac{\varepsilon}{2M})\rt\}$ and let $M$ be an upper bound for $f$ and $g$. Then,
    \begin{align}
        \abs{(fg)(x)-(fg)(y)} &= \abs{f(x)g(x)-f(x)g(y)+f(y)g(x)-f(y)g(y)} \\
            &= \abs{f(x)g(x)-f(x)g(y)}+\abs{f(y)g(x)-f(y)g(y)} \\
            &= \abs{f(x)}\abs{g(x)-g(y)}+\abs{g(y)}\abs{f(x)-f(y)} \\
            &\leq M\abs{g(x)-g(y)}+M\abs{f(x)-f(y)} \\
            &\leq M\frac{\veps}{2M}+M\frac{\veps}{2M} \\
            &\leq \frac{\varepsilon}{2}+\frac{\varepsilon}{2} = \varepsilon
    \end{align}

\end{document}