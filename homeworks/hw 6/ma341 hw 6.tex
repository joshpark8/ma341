\documentclass[11pt]{article}

% Packages
\usepackage[margin=1in]{geometry}
\usepackage{amsfonts}
\usepackage{amsmath}
\usepackage{amssymb}
\usepackage{enumitem}
\usepackage{fleqn}
\usepackage{float}
\usepackage{graphicx}
\usepackage{latexsym}
\usepackage{systeme}

% Fundamental sets
\newcommand{\N}{\mathbb{N}}  % Natural
\newcommand{\R}{\mathbb{R}}  % Real
\newcommand{\C}{\mathbb{C}}  % Complex
\newcommand{\Q}{\mathbb{Q}}  % Rational
\newcommand{\Z}{\mathbb{Z}}  % Integers

% Misc helpers
\setlength{\parskip}{1em}

\newcommand{\eps}{\varepsilon}
\newcommand{\xn}{(x_n)}
\newcommand{\xnk}{(x_{n_k})}

% Title
\author{Josh Park}
\date{\vspace{-1em}Spring 2024}
\title{MA 34100 Homework 6\vspace{-1em}}

% Document
\begin{document}
    \maketitle
    \section*{Exercise 3.3.3}
    Since $x_k\geq 2$,
    \[ x_{k+1} = 1 + \sqrt{x_k-1}\geq 1+\sqrt{2-1}=2\]
    So, $x_n\geq 2$ for all $n\in\N$ by induction. \\
    If $x_{k+1}\leq x_k$,
    \[x_{k+2}=1+\sqrt{x_{k+1}-1}\leq 1+\sqrt{x_k-1}=x_{k+1}\]
    So $\xn$ is decreasing. $x=1$ is not possible, so $\lim \xn = 2$

    \section*{Exercise 3.3.6}
    Case 1: $z_1 \geq \sqrt{a+z_1}$. Then $z_2=\sqrt{a+z_1}\geq z_1$, so the sequence is monotonically increasing.
    \[z*=\frac{1+\sqrt{1+4a}}{2}\]
    Case 2:
    This case contradicts the initial assumption since $z_1>0$ and $a>0$, leading to a positive square root. So, the sequence must be increasing.
you quit you pussy

    \section*{Exercise 3.3.9}

    \section*{Exercise 3.4.1}
    \section*{Exercise 3.4.4}
    \section*{Exercise 3.4.14}

    \section*{Exercise 3.5.2}
    \section*{Exercise 3.5.3}
    \section*{Exercise 3.5.7}
\end{document}